%% start of file `template.tex'.
%% Copyright 2006-2015 Xavier Danaux (xdanaux@gmail.com).
%
% This work may be distributed and/or modified under the
% conditions of the LaTeX Project Public License version 1.3c,
% available at http://www.latex-project.org/lppl/.


\documentclass[11pt,letterpaper,sans]{moderncv}        % possible options include font size ('10pt', '11pt' and '12pt'), paper size ('a4paper', 'letterpaper', 'a5paper', 'legalpaper', 'executivepaper' and 'landscape') and font family ('sans' and 'roman')

\makeatletter
\renewcommand\@biblabel[1]{}
\makeatother
\newcommand\myitem{\item[\textbullet]\hspace*{.5em} }

\usepackage[utf8]{inputenc}

% moderncv themes
\moderncvstyle{banking}                             % style options are 'casual' (default), 'classic', 'banking', 'oldstyle' and 'fancy'
\moderncvcolor{blue}                               % color options 'black', 'blue' (default), 'burgundy', 'green', 'grey', 'orange', 'purple' and 'red'
%\renewcommand{\familydefault}{\sfdefault}         % to set the default font; use '\sfdefault' for the default sans serif font, '\rmdefault' for the default roman one, or any tex font name
%\nopagenumbers{}                                  % uncomment to suppress automatic page numbering for CVs longer than one page

% character encoding
%\usepackage[utf8]{inputenc}                       % if you are not using xelatex ou lualatex, replace by the encoding you are using
%\usepackage{CJKutf8}                              % if you need to use CJK to typeset your resume in Chinese, Japanese or Korean

% adjust the page margins
\usepackage[scale=0.75]{geometry}
%\setlength{\hintscolumnwidth}{3cm}                % if you want to change the width of the column with the dates
%\setlength{\makecvtitlenamewidth}{10cm}           % for the 'classic' style, if you want to force the width allocated to your name and avoid line breaks. be careful though, the length is normally calculated to avoid any overlap with your personal info; use this at your own typographical risks...

% personal data
\name{Peter W.}{Deutsch}
%\title{Resumé title}                               % optional, remove / comment the line if not wanted
\address{Cambridge}{Massachusetts}% optional, remove / comment the line if not wanted; the "postcode city" and "country" arguments can be omitted or provided empty
%\phone[mobile]{+1~(234)~567~890}                   % optional, remove / comment the line if not wanted; the optional "type" of the phone can be "mobile" (default), "fixed" or "fax"
\phone[fixed]{+1~(617)~230~1114}
%\phone[fax]{+3~(456)~789~012}
\email{pwd@mit.edu}                               % optional, remove / comment the line if not wanted
%\homepage{peterdeutsch.ca}                         % optional, remove / comment the line if not wanted
\social[linkedin]{pwdeutsch}                        % optional, remove / comment the line if not wanted
%\social[twitter]{jdoe}                             % optional, remove / comment the line if not wanted
%\social[github]{jdoe}                              % optional, remove / comment the line if not wanted
%\extrainfo{additional information}                 % optional, remove / comment the line if not wanted
% \photo[64pt][0.4pt]{picture}                       % optional, remove / comment the line if not wanted; '64pt' is the height the picture must be resized to, 0.4pt is the thickness of the frame around it (put it to 0pt for no frame) and 'picture' is the name of the picture file
%\quote{Some quote}                                 % optional, remove / comment the line if not wanted

% bibliography adjustements (only useful if you make citations in your resume, or print a list of publications using BibTeX)
%   to show numerical labels in the bibliography (default is to show no labels)
\makeatletter\renewcommand*{\bibliographyitemlabel}{\@biblabel{\arabic{enumiv}}}\makeatother
%   to redefine the bibliography heading string ("Publications")
%\renewcommand{\refname}{Articles}

% bibliography with mutiple entries
%\usepackage{multibib}
%\newcites{book,misc}{{Books},{Others}}
%----------------------------------------------------------------------------------
%            content
%----------------------------------------------------------------------------------
\begin{document}
%\begin{CJK*}{UTF8}{gbsn}                          % to typeset your resume in Chinese using CJK
%-----       resume       ---------------------------------------------------------
\vspace*{-10mm}
\makecvtitle
\vspace*{-10mm}
\section{Education}
\cventry{2022 -- Present}{PhD Student, Electrical Engineering and Computer Science}{Massachusetts Institute of Technology}{}{}{Doctoral Supervisor: Prof. Mengjia Yan}
\cventry{2020 -- 2022}{Master of Science, Electrical Engineering and Computer Science}{Massachusetts Institute of Technology}{}{}{Thesis: Mitigating Memory Controller Side-Channels\\Masters Supervisor: Prof. Mengjia Yan}
\cventry{2014 -- 2020}{Bachelor of Applied Science, Computer Engineering}{University of British Columbia}{}{}{Undergraduate Supervisors: Prof. Mieszko Lis \& Prof. Prashant Nair}  % arguments 3 to 6 can be left empty

\nocite{*}
\bibliographystyle{unsrt}
\bibliography{publications} 
\nobreak

\section{Invited Talks}
\cventry{}{Advanced Micro Devices (AMD) - Architecting for Marginal Defects Working Group}{Machine Learning Accelerator Reliability: Assessing Faults During Training}{May 2025}{}{}{}
\cventry{}{Advanced Micro Devices (AMD) - Architecting for Marginal Defects Working Group}{DelayAVF: Calculating Architectural Vulnerability Factors for Delay Faults}{May 2024}{}{}{}
\cventry{}{MIT AI Hardware Program Symposium}{Increasing Architectural Resilience to Small Delay Faults}{May 2024}{}{}{}
\cventry{}{MIT CSAIL/UK GHCQ Delegation}{Strengthening Hardware Security in the Age of AI}{Sept. 2023}{}{}{}
\cventry{}{Carnegie Mellon University - Special Topics in Hardware Security (17-715)}{Modelling Obfuscating Side-Channel Defense Schemes}{Sept. 2023}{}{}{}

\section{Academic Service}
\cventry{}{External Review Committee Member}{IEEE International Symposium on High-Performance Computer Architecture (HPCA)}{2025}{}{}{}
\cventry{}{Reviewer}{IEEE Transactions on Computers – Special Issue on Hardware Security}{2022}{}{}{}

\section{Work Experience}
\subsection{Research \& Academic}
\cventry{2025}{PhD Intern}{Meta}{Sunnyvale, CA}{}{%
  \begin{itemize}\setlength\itemindent{-8pt}
  \setlength\labelsep{-3pt}
      \myitem Analyzed long-term silent data corruption (SDC) behaviours across Meta's CPU fleet.
      \myitem Developed reliability forecasting models for hyper-scale computing platforms, helping to scale up next-generation LLM training.
  \end{itemize}}
\cventry{2024}{PhD Intern}{Advanced Micro Devices (AMD)}{Austin, TX}{}{%
  \begin{itemize}\setlength\itemindent{-8pt}
  \setlength\labelsep{-3pt}
      \myitem Applied my prior research (DelayAVF) to assess the potential reliability impacts of small delay faults in next-generation AMD products.
      \myitem Substantially expanded upon the existing open-source DelayAVF infrastructure to facilitate the reliability assessment of large out-of-order CPUs.
  \end{itemize}}
\cventry{2022 -- 2024}{Teaching Assistant/Lab Assignment Developer}{Massachusetts Institute of Technology}{Cambridge, MA}{}{%
  \begin{itemize}\setlength\itemindent{-8pt}
  \setlength\labelsep{-3pt}
      \myitem Assisted in the development and deployment of lab assignments for MIT's Secure Hardware Design course.
      \myitem Developed an assignment which guides students through performing and characterizing Rowhammer attacks on commodity hardware.
  \end{itemize}}

\cventry{2019 -- 2020}{Undergraduate Research Student}{University of British Columbia}{Vancouver, Canada}{}{%
  \begin{itemize}\setlength\itemindent{-8pt}
  \setlength\labelsep{-3pt}
      \myitem Investigated methods to detect and mitigate speculative execution attacks which utilize cache and DRAM side-channels (ex. Spectre/Meltdown).
      \myitem Replicated attacks, benchmarked prior work, and explored new mitigations using SPEC CPU 2017 and gem5.
  \end{itemize}}
  
\cventry{2017}{Microsystems Engineering Student}{Bosch Corporate Research}{Stuttgart, Germany}{}{
  \begin{itemize}\setlength\itemindent{-8.5pt}
  \setlength\labelsep{-3pt}
      \myitem Researched the use of MEMS gyroscopes as Physical Unclonable Functions (PUFs), facilitating reliable secret key generation in IoT devices.
      \myitem Helped to devise and evaluate entropy extraction schemes to generate cryptographically secure keys from highly correlated device features.
  \end{itemize}}

\cventry{2016 -- 2020}{Undergraduate Teaching Assistant}{University of British Columbia}{Vancouver, Canada}{}{
  \begin{itemize}\setlength\itemindent{-8.5pt}
  \setlength\labelsep{-3pt}
      \myitem Conveyed Verilog-focused digital design content to hundreds of second and third-year undergraduate students.
      \myitem Taught CPEN 211 (Introduction to Microcomputers), CPEN 311 (Digital Systems Design), and CPEN 391 (Computer Engineering Design Studio II). 
  \end{itemize}}

\subsection{Industry}
\cventry{2018 -- 2019}{Verification Engineer Intern}{Intel Corporation}{Vancouver, Canada}{}
{%
  \begin{itemize}\setlength\itemindent{-8.5pt}
  \setlength\labelsep{-3pt}
      \myitem Verified system controller ASICs for Intel NAND devices using SystemVerilog and the Universal Verification Methodology (UVM).
      \myitem Designed end-to-end traffic tests to confirm compliance to internal architecture requirements and flash interface specifications, ensuring that comprehensive code coverage was achieved.
  \end{itemize}}

\cventry{2017}{Product Design Engineer Intern}{Microsemi (Microchip)}{Vancouver, Canada}{}{%
  \begin{itemize}\setlength\itemindent{-8.5pt}
  \setlength\labelsep{-3pt}
      \myitem Designed and verified top-level RTL glue logic (SystemVerilog \& VHDL) for SAS/SATA RAID controllers.
      \myitem Implemented appropriate pipelining and clock-domain-crossing synchronization strategies, ensuring that timing closure and MTBF thresholds were met.
  \end{itemize}}

\section{Awards}
\subsection{Graduate}
\cventry{}{Awarded to papers deemed the most significant in computer architecture research in 2024}{IEEE Micro Top Picks Honourable Mention (DelayAVF)}{2025}{}{}
\cventry{}{Topic: Leveraging Accessible Signals for the Efficient Discovery of Corrupt Execution Errors}{Google Research Scholar Grant}{2023}{}{}
\cventry{}{Awarded on the recommendation of the Department Head of EECS}{Advanced Televison and Signal Processing Fellowship}{2020}{}{}{}
\subsection{Undergraduate}
\cventry{}{Awarded to the head of the graduating undergraduate class in Applied Science}{Dean’s Prize for Academic Excellence in Engineering}{2020}{}{}{}
\cventry{}{Presented to the top ECE Capstone (final year) project teams in 2020}{ECE Capstone Faculty Award}{2020}{}{}{}
\cventry{}{Awarded on the recommendation of the Faculty of Applied Science}{NSERC Undergraduate Student Research Award}{2019}{}{}{}
\cventry{}{Awarded to students in the top 5\% of their program}{Trek Excellence Scholarship for Continuing Students}{2015, 2016, 2017, 2019}{}{}{}
\cventry{}{Awarded on the recommendation of the Department Head of Computer Engineering}{PMC-Sierra Founders Award in Electrical and Computer Engineering}{2019}{}{}{}
\cventry{}{Awarded on the recommendation of the Faculty of Applied Science}{Elizabeth and Leslie Gould Scholarship in Engineering}{2019}{}{}{}
\cventry{}{Awarded on the recommendation of the Faculty of Applied Science}{J Fred Muir Memorial Scholarship in Engineering}{2017}{}{}{}
\cventry{}{Awarded on the recommendation of the Faculty of Applied Science}{J K Zee Memorial Scholarship}{2016}{}{}{}


\section{Volunteerism}
\cventry{2021 -- Present}{Treasurer/Graduate Student Volunteer}{MIT Graduate Application Assistance Program}{Cambridge, MA}{}{%
  \begin{itemize}\setlength\itemindent{-8pt}
  \setlength\labelsep{-3pt}
       \myitem Worked with underrepresented  MIT PhD applicants, providing advice and detailed feedback on personal and research statements. 
       \myitem Coordinated finances for the program, raising funds to provide fee waivers for underprivileged applicants.
  \end{itemize}}
\cventry{2020}{Printing / Distribution Volunteer}{BC COVID-19 3D Printing Group (BCC3D)}{Vancouver, Canada}{}{%
  \begin{itemize}\setlength\itemindent{-8pt}
  \setlength\labelsep{-3pt}
       \myitem Personally manufactured 300+ 3D printed face shield visors and 'ear savers' for use at hospitals and clinics. 
       \myitem Inspected, sanitized, and packed 10,000+ articles of PPE produced by local volunteers. 
  \end{itemize}}
  


%\section{References}
%\cvitemwithcomment{\textbf{Asst. Professor Mieszko Lis}}{Research Supervisor}{University of British Columbia}
%\cvitemwithcomment{\textbf{Asst. Professor Prashant Nair}}{Research Supervisor \& Instructor}{University of British Columbia}
%\cvitemwithcomment{\textbf{Professor Tor Aamodt}}{TA Supervisor \& Former Instructor}{University of British Columbia}
%\cvitemwithcomment{\textbf{Dr. Oliver Willers}}{Former Research Supervisor}{Robert Bosch GmbH}

\end{document}
